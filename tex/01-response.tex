%! TEX root=../main.tex

\section{Responses}
  \subsection{Response 1}
    \begin{quotation}
      Humans have many forms of emotion in them and nothing can turn on why
        humans are so emotional, especially women. It is evident from research
        that women are more emotional than men who use logic more than their
        emotions. New research by McDuff, Kodra, Kaliouby, \& LaFrance suggests
        they are. And they aren’t. Women do smile more than men, and there is
        evidence that women exaggerate facial expressions for positive emotion.
        However, McDuff and colleagues believe that smiling and other displays
        of positive emotion are only part of the picture. Emotions can be
        negative as well as positive, and within each valence there are a range
        of distinct emotional states, including fear, disgust, anger, joy,
        satisfaction, and gratitude.

      It can be seen from what happened in the video of Brooke and Gary in the
        middle of an argument that from small to large. Sometimes when fighting
        with someone we already know closely, we often bring up problems that we
        shouldn't have in any relationship it is not good to bury feelings and
        burdens in oneself. From the dialogue in the video, Brooke often
        comments on Gary because of his behavior at home, from his habits to
        Gary's physique. Indirectly, Brooke only conveyed his feelings that he
        did not want to keep because she could burden the life of this couple,
        but from Gary's view, he saw that Brooke did not give him space to rest
        from work, even though he worked so that he could support their
        household, but what that Brooke said when he got home was just babbling
        and complaining.

      In the CRKit the solution to this problem is not escape but number 6,
        namely managing emotions. In a better relationship there are students
        who give in, there's no need to raise problems. Judging from what
        Brooke said there was a time when Gary took her to a place she didn't
        like, but because there was Gay she still went. Only because Gary didn't
        like ballet he decided not to go with Brooke's whim which was wrong.
        Expressing opinions and expressions is never wrong, what responders
        need is their openness to comments and suggestions that others have
        given to them. Sometimes the fault comes from the person providing
        comments and suggestions, they use emotion and pressure in speaking to
        stir up the other person's emotions.
    \end{quotation}

    \paragraph{This is a response to Marcella Ferchette on Post ID 43333650}
      Great post, Marcella! I especially enjoyed your analysis in your second
        paragraph where you discussed the nasty things people tend to say when
        fighting with those close to us. During the argument between Gary and
        Brooke, we definitely did not see any focus on one specific issue one
        person may have had with the other. Instead, we see them attempt to
        \textit{tear each other apart} on a \textit{person} and
        \textit{character}-based level. This video was incredibly painful to
        watch and hit far too close to home, I am sure, for a lot of us!
