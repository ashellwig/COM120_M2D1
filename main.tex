\documentclass[stu,12pt]{apa7}
  \usepackage{times}               % Times New Roman Font Face
  \usepackage[american]{babel}      % Localization
  \usepackage[utf8]{inputenc}      % Input Encoding
  \usepackage{hyperref}            % Hyperlinks
  \usepackage{enumitem}            % Additional Enumeration Environment Settings
  \usepackage{geometry}            % Page Layout
  \usepackage{soul}                % Text Highlighting
  \usepackage{graphicx}            % Images
  \usepackage{csquotes}            % Quoting Environment
  \usepackage{bookmark}            % Required by `csquotes'
  \usepackage{mdframed}            % Colorful Tex-Box Environment
  \usepackage[toc]{appendix}       % Appendix
  \usepackage{fancyhdr}            % Headings and Footers
  \usepackage[%
    style=apa,%
    sortcites=true,%
    sorting=nyt%
  ]{biblatex}
  \usepackage{xcolor}

  % Bibliography Setup
  %% Language Mappings
  \DeclareLanguageMapping{english}{english-apa}
  \DeclareLanguageMapping{american}{american-apa}
  %% Bibliography File Path
  \addbibresource{main.bib}
  %% Categories for Specified Bibliography Items
  %%% Category for sources not referenced in-text
  \DeclareBibliographyCategory{consulted}
  \addtocategory{consulted}{noauthor_communication_2013}
  \addtocategory{consulted}{noauthor_business_nodate}
  \addtocategory{consulted}{noauthor_college_nodate}
  \addtocategory{consulted}{thomas_exploration_2020}
  \addtocategory{consulted}{noauthor_he_nodate}
  \addtocategory{consulted}{reed_leaving_2006}
  \addtocategory{consulted}{bacha_how_2016}


  % Hyperlink Setup
  \hypersetup{
    colorlinks = true,
    urlcolor = blue,
    linkcolor = blue,
    citecolor = blue
  }

  % Page and Text Layout
  \geometry{%
    a4paper,%
    top=1in,%
    bottom=1in,%
    left=1in,%
    right=1in%
  }
  \setlength{\headheight}{15pt}

  % Header
  \lhead{COM120CG1-M2D1}

  % Title Page
  \title{%
    M2D1: Fight or Flight
  }
  \shorttitle{Module 2 Discussion 1}
  \author{Ashton Hellwig}
  \authorsaffiliations{Department of Mathematics, Front Range Community College}
  \course{COM115: Interpersonal Communication}
  \professor{Richard Thomas}
  \duedate{November 14, 2020 23:59:59 MDT}
  \date{\today}
  \abstract{%
    \textbf{Overview}\\%
    We all deal with conflict differently. Some of us avoid an argument, while
      others escalate it. Perhaps you are willing to compromise, or maybe you
      would rather compete. As we saw in the movie clip, having different
      conflict resolution styles can make it very hard to reach an agreement.
      Even if you never change your conflict resolution method, being able to
      identify its strengths and weaknesses will help you become a better
      communicator.\\%

    You should spend approximately 4 hours on this assignment.%
  }

\begin{document}
  % Title Page
  \maketitle

  % Initial Post
  \section{Initial Post}
    \subsection*{Instructions}
      \begin{enumerate}
        \item Be sure you have completed your readings and explored all the
          materials, including videos, in the Exploration page accessible within
          the ``Read/View'' topic of Module 2 Content.
        \item In the video appearing within the topic titled, ``Gender and
          Conflict'' we saw that Brooke and Gary had two very different ways of
          dealing with conflict, which caused a small disagreement to escalate
          into a huge argument.
        \item Click on the ``CR Kit'' Word Document link appearing on the fourth
          topic titled ``Improving Your Skills'' which is listed within
          ``Exploration of He Said! She Said'' under ``Read/Review''. Next,
          review the self-assessment content to determine your personal conflict
          resolution method (style).
          \begin{itemize}
            \item \textbf{NOTE}: There is NOT an assessment for you to complete.
              Again, REVIEW the self-assessment CONTENT to determine your
              personal conflict resolution method (style). If you find yourself
              equally split between two or more methods, then choose the one you
              feel is most important to you.
          \end{itemize}
        \item Conduct research of the conflict resolution method (style) you
          selected. For example what are the pros and cons of the conflict
          resolution method (style) based on your research?
      \end{enumerate}


      % My Conflict Resolution Method/Style
      \newpage
      \paragraph{My Conflict Resolution Method: Empathy}
        The conflict resolution method I tend to lean towards the most would
          have to be that of empathy. Empathy goes a \textit{long} way when
          resolving conflict. Many times throughout the duration of an argument,
          one party may not be feeling \emph{heard}. The main theme in empathy
          is a high degree of being able to understand the other party's
          feelings, and allowing them to feel heard
          \parencite[pp. 48]{wied_empathy_2007}. Empathy is not only a
          pleasurable experience for the speaker due to a feeling of
          understanding, but also for the listener as it can allow them to see
          the conflict from another point of view, leading to a quicker conflict
          resolution timeline \parencite[pp. 49]{wied_empathy_2007}.

      % The Pitfalls of Empathy
      \paragraph{The Pitfalls of Empathy}
        While empathy does tend to be a safe route to take in terms of conflict
          resolution, there are a few negatives to having a high degree of
          empathy. ``Empaths'' (Those with a high degree of empathy) can be
          overwhelmed in situations of intense emotion and even actually
          \textit{feel} the other person's pain
          \parencite[pp. 376]{schieman_when_2001}. Sometimes, this could be
          interpreted by the other individual as a way for the listener to
          ``make a situation about them'', rather than associating physical
          empathy with sympathy. Furthermore, those with a higher level of
          empathy tend to also have higher instances of symptoms of depression
          \parencite[pp. 376]{schieman_when_2001}. Finally, empathy is a hard
          metric for one to measure, as there are a multitude of systems
          designed for the academic reporting of empathy in academic studies
          \parencite[pp. 675]{zaki_neuroscience_2012}.


  % Replies
  % %! TEX root=../main.tex

\section{Responses}
  \subsection{Response 1}
    \begin{quotation}
      Humans have many forms of emotion in them and nothing can turn on why
        humans are so emotional, especially women. It is evident from research
        that women are more emotional than men who use logic more than their
        emotions. New research by McDuff, Kodra, Kaliouby, \& LaFrance suggests
        they are. And they aren’t. Women do smile more than men, and there is
        evidence that women exaggerate facial expressions for positive emotion.
        However, McDuff and colleagues believe that smiling and other displays
        of positive emotion are only part of the picture. Emotions can be
        negative as well as positive, and within each valence there are a range
        of distinct emotional states, including fear, disgust, anger, joy,
        satisfaction, and gratitude.

      It can be seen from what happened in the video of Brooke and Gary in the
        middle of an argument that from small to large. Sometimes when fighting
        with someone we already know closely, we often bring up problems that we
        shouldn't have in any relationship it is not good to bury feelings and
        burdens in oneself. From the dialogue in the video, Brooke often
        comments on Gary because of his behavior at home, from his habits to
        Gary's physique. Indirectly, Brooke only conveyed his feelings that he
        did not want to keep because she could burden the life of this couple,
        but from Gary's view, he saw that Brooke did not give him space to rest
        from work, even though he worked so that he could support their
        household, but what that Brooke said when he got home was just babbling
        and complaining.

      In the CRKit the solution to this problem is not escape but number 6,
        namely managing emotions. In a better relationship there are students
        who give in, there's no need to raise problems. Judging from what
        Brooke said there was a time when Gary took her to a place she didn't
        like, but because there was Gay she still went. Only because Gary didn't
        like ballet he decided not to go with Brooke's whim which was wrong.
        Expressing opinions and expressions is never wrong, what responders
        need is their openness to comments and suggestions that others have
        given to them. Sometimes the fault comes from the person providing
        comments and suggestions, they use emotion and pressure in speaking to
        stir up the other person's emotions.
    \end{quotation}

    \paragraph{This is a response to Marcella Ferchette on Post ID 43333650}
      Great post, Marcella! I especially enjoyed your analysis in your second
        paragraph where you discussed the nasty things people tend to say when
        fighting with those close to us. During the argument between Gary and
        Brooke, we definitely did not see any focus on one specific issue one
        person may have had with the other. Instead, we see them attempt to
        \textit{tear each other apart} on a \textit{person} and
        \textit{character}-based level. This video was incredibly painful to
        watch and hit far too close to home, I am sure, for a lot of us!

  % %! TEX root=../main.tex

\subsection{Response 2}
  \begin{quotation}
    My favorite form of conflict resolution is collaboration. It is rarely used
      systematically in society itself, but it is incredible in the workplace.
      In collaboration we build relationships with other people and utilize
      the strengths and weaknesses of one another. The team uses partners from
      outside the collaboration such as industry partners to make a powerful
      decision the team can have all individual inputs.

    Collaboration can be effective because the return on investment in time can
      be maximized. In a good collaboration there is a massive amount of
      research and discussion completed. There should not be a lot of voting
      because that will limit the decision making.

    In the current situation with COVID-19 where colleagues and team members
      are in completely different physical locations at an unprecedented level,
      it is even more essential to build collaborative skills to solve conflict.
  \end{quotation}

  \paragraph{This is a response to Cherylee Parker on Post ID 43279232}
    Interesting analysis, Cherylee! I agree with you on the fact that
      collaboration is important in a workplace environment. Have you ever been
      a part of a workplace environment where collaboration is difficult, if
      not impossible, with your fellow coworkers (or, usually, management)?
      I know I personally have, and perhaps that is why I find work to be an
      even more difficult place to maintain solid foundations of conflict
      resolution when you just \textit{know} you are the one in the right. Have
      you any tips/tricks/techniques to get around this in your own experience?


  % Bibliography
  %% Works Cited
  \newpage
  \printbibliography[%
    title={References},%
    heading={bibintoc},%
    notcategory={consulted}%
  ]

  %% Works Consulted
  \newpage
  \nocite{*}
  \printbibliography[%
    title={Additional References},%
    heading={bibintoc},%
    category={consulted}%
  ]
\end{document}
